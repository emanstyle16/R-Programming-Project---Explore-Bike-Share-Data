
% Default to the notebook output style

    


% Inherit from the specified cell style.




    
\documentclass[11pt]{article}

    
    
    \usepackage[T1]{fontenc}
    % Nicer default font (+ math font) than Computer Modern for most use cases
    \usepackage{mathpazo}

    % Basic figure setup, for now with no caption control since it's done
    % automatically by Pandoc (which extracts ![](path) syntax from Markdown).
    \usepackage{graphicx}
    % We will generate all images so they have a width \maxwidth. This means
    % that they will get their normal width if they fit onto the page, but
    % are scaled down if they would overflow the margins.
    \makeatletter
    \def\maxwidth{\ifdim\Gin@nat@width>\linewidth\linewidth
    \else\Gin@nat@width\fi}
    \makeatother
    \let\Oldincludegraphics\includegraphics
    % Set max figure width to be 80% of text width, for now hardcoded.
    \renewcommand{\includegraphics}[1]{\Oldincludegraphics[width=.8\maxwidth]{#1}}
    % Ensure that by default, figures have no caption (until we provide a
    % proper Figure object with a Caption API and a way to capture that
    % in the conversion process - todo).
    \usepackage{caption}
    \DeclareCaptionLabelFormat{nolabel}{}
    \captionsetup{labelformat=nolabel}

    \usepackage{adjustbox} % Used to constrain images to a maximum size 
    \usepackage{xcolor} % Allow colors to be defined
    \usepackage{enumerate} % Needed for markdown enumerations to work
    \usepackage{geometry} % Used to adjust the document margins
    \usepackage{amsmath} % Equations
    \usepackage{amssymb} % Equations
    \usepackage{textcomp} % defines textquotesingle
    % Hack from http://tex.stackexchange.com/a/47451/13684:
    \AtBeginDocument{%
        \def\PYZsq{\textquotesingle}% Upright quotes in Pygmentized code
    }
    \usepackage{upquote} % Upright quotes for verbatim code
    \usepackage{eurosym} % defines \euro
    \usepackage[mathletters]{ucs} % Extended unicode (utf-8) support
    \usepackage[utf8x]{inputenc} % Allow utf-8 characters in the tex document
    \usepackage{fancyvrb} % verbatim replacement that allows latex
    \usepackage{grffile} % extends the file name processing of package graphics 
                         % to support a larger range 
    % The hyperref package gives us a pdf with properly built
    % internal navigation ('pdf bookmarks' for the table of contents,
    % internal cross-reference links, web links for URLs, etc.)
    \usepackage{hyperref}
    \usepackage{longtable} % longtable support required by pandoc >1.10
    \usepackage{booktabs}  % table support for pandoc > 1.12.2
    \usepackage[inline]{enumitem} % IRkernel/repr support (it uses the enumerate* environment)
    \usepackage[normalem]{ulem} % ulem is needed to support strikethroughs (\sout)
                                % normalem makes italics be italics, not underlines
    

    
    
    % Colors for the hyperref package
    \definecolor{urlcolor}{rgb}{0,.145,.698}
    \definecolor{linkcolor}{rgb}{.71,0.21,0.01}
    \definecolor{citecolor}{rgb}{.12,.54,.11}

    % ANSI colors
    \definecolor{ansi-black}{HTML}{3E424D}
    \definecolor{ansi-black-intense}{HTML}{282C36}
    \definecolor{ansi-red}{HTML}{E75C58}
    \definecolor{ansi-red-intense}{HTML}{B22B31}
    \definecolor{ansi-green}{HTML}{00A250}
    \definecolor{ansi-green-intense}{HTML}{007427}
    \definecolor{ansi-yellow}{HTML}{DDB62B}
    \definecolor{ansi-yellow-intense}{HTML}{B27D12}
    \definecolor{ansi-blue}{HTML}{208FFB}
    \definecolor{ansi-blue-intense}{HTML}{0065CA}
    \definecolor{ansi-magenta}{HTML}{D160C4}
    \definecolor{ansi-magenta-intense}{HTML}{A03196}
    \definecolor{ansi-cyan}{HTML}{60C6C8}
    \definecolor{ansi-cyan-intense}{HTML}{258F8F}
    \definecolor{ansi-white}{HTML}{C5C1B4}
    \definecolor{ansi-white-intense}{HTML}{A1A6B2}

    % commands and environments needed by pandoc snippets
    % extracted from the output of `pandoc -s`
    \providecommand{\tightlist}{%
      \setlength{\itemsep}{0pt}\setlength{\parskip}{0pt}}
    \DefineVerbatimEnvironment{Highlighting}{Verbatim}{commandchars=\\\{\}}
    % Add ',fontsize=\small' for more characters per line
    \newenvironment{Shaded}{}{}
    \newcommand{\KeywordTok}[1]{\textcolor[rgb]{0.00,0.44,0.13}{\textbf{{#1}}}}
    \newcommand{\DataTypeTok}[1]{\textcolor[rgb]{0.56,0.13,0.00}{{#1}}}
    \newcommand{\DecValTok}[1]{\textcolor[rgb]{0.25,0.63,0.44}{{#1}}}
    \newcommand{\BaseNTok}[1]{\textcolor[rgb]{0.25,0.63,0.44}{{#1}}}
    \newcommand{\FloatTok}[1]{\textcolor[rgb]{0.25,0.63,0.44}{{#1}}}
    \newcommand{\CharTok}[1]{\textcolor[rgb]{0.25,0.44,0.63}{{#1}}}
    \newcommand{\StringTok}[1]{\textcolor[rgb]{0.25,0.44,0.63}{{#1}}}
    \newcommand{\CommentTok}[1]{\textcolor[rgb]{0.38,0.63,0.69}{\textit{{#1}}}}
    \newcommand{\OtherTok}[1]{\textcolor[rgb]{0.00,0.44,0.13}{{#1}}}
    \newcommand{\AlertTok}[1]{\textcolor[rgb]{1.00,0.00,0.00}{\textbf{{#1}}}}
    \newcommand{\FunctionTok}[1]{\textcolor[rgb]{0.02,0.16,0.49}{{#1}}}
    \newcommand{\RegionMarkerTok}[1]{{#1}}
    \newcommand{\ErrorTok}[1]{\textcolor[rgb]{1.00,0.00,0.00}{\textbf{{#1}}}}
    \newcommand{\NormalTok}[1]{{#1}}
    
    % Additional commands for more recent versions of Pandoc
    \newcommand{\ConstantTok}[1]{\textcolor[rgb]{0.53,0.00,0.00}{{#1}}}
    \newcommand{\SpecialCharTok}[1]{\textcolor[rgb]{0.25,0.44,0.63}{{#1}}}
    \newcommand{\VerbatimStringTok}[1]{\textcolor[rgb]{0.25,0.44,0.63}{{#1}}}
    \newcommand{\SpecialStringTok}[1]{\textcolor[rgb]{0.73,0.40,0.53}{{#1}}}
    \newcommand{\ImportTok}[1]{{#1}}
    \newcommand{\DocumentationTok}[1]{\textcolor[rgb]{0.73,0.13,0.13}{\textit{{#1}}}}
    \newcommand{\AnnotationTok}[1]{\textcolor[rgb]{0.38,0.63,0.69}{\textbf{\textit{{#1}}}}}
    \newcommand{\CommentVarTok}[1]{\textcolor[rgb]{0.38,0.63,0.69}{\textbf{\textit{{#1}}}}}
    \newcommand{\VariableTok}[1]{\textcolor[rgb]{0.10,0.09,0.49}{{#1}}}
    \newcommand{\ControlFlowTok}[1]{\textcolor[rgb]{0.00,0.44,0.13}{\textbf{{#1}}}}
    \newcommand{\OperatorTok}[1]{\textcolor[rgb]{0.40,0.40,0.40}{{#1}}}
    \newcommand{\BuiltInTok}[1]{{#1}}
    \newcommand{\ExtensionTok}[1]{{#1}}
    \newcommand{\PreprocessorTok}[1]{\textcolor[rgb]{0.74,0.48,0.00}{{#1}}}
    \newcommand{\AttributeTok}[1]{\textcolor[rgb]{0.49,0.56,0.16}{{#1}}}
    \newcommand{\InformationTok}[1]{\textcolor[rgb]{0.38,0.63,0.69}{\textbf{\textit{{#1}}}}}
    \newcommand{\WarningTok}[1]{\textcolor[rgb]{0.38,0.63,0.69}{\textbf{\textit{{#1}}}}}
    
    
    % Define a nice break command that doesn't care if a line doesn't already
    % exist.
    \def\br{\hspace*{\fill} \\* }
    % Math Jax compatability definitions
    \def\gt{>}
    \def\lt{<}
    % Document parameters
    \title{project}
    
    
    

    % Pygments definitions
    
\makeatletter
\def\PY@reset{\let\PY@it=\relax \let\PY@bf=\relax%
    \let\PY@ul=\relax \let\PY@tc=\relax%
    \let\PY@bc=\relax \let\PY@ff=\relax}
\def\PY@tok#1{\csname PY@tok@#1\endcsname}
\def\PY@toks#1+{\ifx\relax#1\empty\else%
    \PY@tok{#1}\expandafter\PY@toks\fi}
\def\PY@do#1{\PY@bc{\PY@tc{\PY@ul{%
    \PY@it{\PY@bf{\PY@ff{#1}}}}}}}
\def\PY#1#2{\PY@reset\PY@toks#1+\relax+\PY@do{#2}}

\expandafter\def\csname PY@tok@w\endcsname{\def\PY@tc##1{\textcolor[rgb]{0.73,0.73,0.73}{##1}}}
\expandafter\def\csname PY@tok@c\endcsname{\let\PY@it=\textit\def\PY@tc##1{\textcolor[rgb]{0.25,0.50,0.50}{##1}}}
\expandafter\def\csname PY@tok@cp\endcsname{\def\PY@tc##1{\textcolor[rgb]{0.74,0.48,0.00}{##1}}}
\expandafter\def\csname PY@tok@k\endcsname{\let\PY@bf=\textbf\def\PY@tc##1{\textcolor[rgb]{0.00,0.50,0.00}{##1}}}
\expandafter\def\csname PY@tok@kp\endcsname{\def\PY@tc##1{\textcolor[rgb]{0.00,0.50,0.00}{##1}}}
\expandafter\def\csname PY@tok@kt\endcsname{\def\PY@tc##1{\textcolor[rgb]{0.69,0.00,0.25}{##1}}}
\expandafter\def\csname PY@tok@o\endcsname{\def\PY@tc##1{\textcolor[rgb]{0.40,0.40,0.40}{##1}}}
\expandafter\def\csname PY@tok@ow\endcsname{\let\PY@bf=\textbf\def\PY@tc##1{\textcolor[rgb]{0.67,0.13,1.00}{##1}}}
\expandafter\def\csname PY@tok@nb\endcsname{\def\PY@tc##1{\textcolor[rgb]{0.00,0.50,0.00}{##1}}}
\expandafter\def\csname PY@tok@nf\endcsname{\def\PY@tc##1{\textcolor[rgb]{0.00,0.00,1.00}{##1}}}
\expandafter\def\csname PY@tok@nc\endcsname{\let\PY@bf=\textbf\def\PY@tc##1{\textcolor[rgb]{0.00,0.00,1.00}{##1}}}
\expandafter\def\csname PY@tok@nn\endcsname{\let\PY@bf=\textbf\def\PY@tc##1{\textcolor[rgb]{0.00,0.00,1.00}{##1}}}
\expandafter\def\csname PY@tok@ne\endcsname{\let\PY@bf=\textbf\def\PY@tc##1{\textcolor[rgb]{0.82,0.25,0.23}{##1}}}
\expandafter\def\csname PY@tok@nv\endcsname{\def\PY@tc##1{\textcolor[rgb]{0.10,0.09,0.49}{##1}}}
\expandafter\def\csname PY@tok@no\endcsname{\def\PY@tc##1{\textcolor[rgb]{0.53,0.00,0.00}{##1}}}
\expandafter\def\csname PY@tok@nl\endcsname{\def\PY@tc##1{\textcolor[rgb]{0.63,0.63,0.00}{##1}}}
\expandafter\def\csname PY@tok@ni\endcsname{\let\PY@bf=\textbf\def\PY@tc##1{\textcolor[rgb]{0.60,0.60,0.60}{##1}}}
\expandafter\def\csname PY@tok@na\endcsname{\def\PY@tc##1{\textcolor[rgb]{0.49,0.56,0.16}{##1}}}
\expandafter\def\csname PY@tok@nt\endcsname{\let\PY@bf=\textbf\def\PY@tc##1{\textcolor[rgb]{0.00,0.50,0.00}{##1}}}
\expandafter\def\csname PY@tok@nd\endcsname{\def\PY@tc##1{\textcolor[rgb]{0.67,0.13,1.00}{##1}}}
\expandafter\def\csname PY@tok@s\endcsname{\def\PY@tc##1{\textcolor[rgb]{0.73,0.13,0.13}{##1}}}
\expandafter\def\csname PY@tok@sd\endcsname{\let\PY@it=\textit\def\PY@tc##1{\textcolor[rgb]{0.73,0.13,0.13}{##1}}}
\expandafter\def\csname PY@tok@si\endcsname{\let\PY@bf=\textbf\def\PY@tc##1{\textcolor[rgb]{0.73,0.40,0.53}{##1}}}
\expandafter\def\csname PY@tok@se\endcsname{\let\PY@bf=\textbf\def\PY@tc##1{\textcolor[rgb]{0.73,0.40,0.13}{##1}}}
\expandafter\def\csname PY@tok@sr\endcsname{\def\PY@tc##1{\textcolor[rgb]{0.73,0.40,0.53}{##1}}}
\expandafter\def\csname PY@tok@ss\endcsname{\def\PY@tc##1{\textcolor[rgb]{0.10,0.09,0.49}{##1}}}
\expandafter\def\csname PY@tok@sx\endcsname{\def\PY@tc##1{\textcolor[rgb]{0.00,0.50,0.00}{##1}}}
\expandafter\def\csname PY@tok@m\endcsname{\def\PY@tc##1{\textcolor[rgb]{0.40,0.40,0.40}{##1}}}
\expandafter\def\csname PY@tok@gh\endcsname{\let\PY@bf=\textbf\def\PY@tc##1{\textcolor[rgb]{0.00,0.00,0.50}{##1}}}
\expandafter\def\csname PY@tok@gu\endcsname{\let\PY@bf=\textbf\def\PY@tc##1{\textcolor[rgb]{0.50,0.00,0.50}{##1}}}
\expandafter\def\csname PY@tok@gd\endcsname{\def\PY@tc##1{\textcolor[rgb]{0.63,0.00,0.00}{##1}}}
\expandafter\def\csname PY@tok@gi\endcsname{\def\PY@tc##1{\textcolor[rgb]{0.00,0.63,0.00}{##1}}}
\expandafter\def\csname PY@tok@gr\endcsname{\def\PY@tc##1{\textcolor[rgb]{1.00,0.00,0.00}{##1}}}
\expandafter\def\csname PY@tok@ge\endcsname{\let\PY@it=\textit}
\expandafter\def\csname PY@tok@gs\endcsname{\let\PY@bf=\textbf}
\expandafter\def\csname PY@tok@gp\endcsname{\let\PY@bf=\textbf\def\PY@tc##1{\textcolor[rgb]{0.00,0.00,0.50}{##1}}}
\expandafter\def\csname PY@tok@go\endcsname{\def\PY@tc##1{\textcolor[rgb]{0.53,0.53,0.53}{##1}}}
\expandafter\def\csname PY@tok@gt\endcsname{\def\PY@tc##1{\textcolor[rgb]{0.00,0.27,0.87}{##1}}}
\expandafter\def\csname PY@tok@err\endcsname{\def\PY@bc##1{\setlength{\fboxsep}{0pt}\fcolorbox[rgb]{1.00,0.00,0.00}{1,1,1}{\strut ##1}}}
\expandafter\def\csname PY@tok@kc\endcsname{\let\PY@bf=\textbf\def\PY@tc##1{\textcolor[rgb]{0.00,0.50,0.00}{##1}}}
\expandafter\def\csname PY@tok@kd\endcsname{\let\PY@bf=\textbf\def\PY@tc##1{\textcolor[rgb]{0.00,0.50,0.00}{##1}}}
\expandafter\def\csname PY@tok@kn\endcsname{\let\PY@bf=\textbf\def\PY@tc##1{\textcolor[rgb]{0.00,0.50,0.00}{##1}}}
\expandafter\def\csname PY@tok@kr\endcsname{\let\PY@bf=\textbf\def\PY@tc##1{\textcolor[rgb]{0.00,0.50,0.00}{##1}}}
\expandafter\def\csname PY@tok@bp\endcsname{\def\PY@tc##1{\textcolor[rgb]{0.00,0.50,0.00}{##1}}}
\expandafter\def\csname PY@tok@fm\endcsname{\def\PY@tc##1{\textcolor[rgb]{0.00,0.00,1.00}{##1}}}
\expandafter\def\csname PY@tok@vc\endcsname{\def\PY@tc##1{\textcolor[rgb]{0.10,0.09,0.49}{##1}}}
\expandafter\def\csname PY@tok@vg\endcsname{\def\PY@tc##1{\textcolor[rgb]{0.10,0.09,0.49}{##1}}}
\expandafter\def\csname PY@tok@vi\endcsname{\def\PY@tc##1{\textcolor[rgb]{0.10,0.09,0.49}{##1}}}
\expandafter\def\csname PY@tok@vm\endcsname{\def\PY@tc##1{\textcolor[rgb]{0.10,0.09,0.49}{##1}}}
\expandafter\def\csname PY@tok@sa\endcsname{\def\PY@tc##1{\textcolor[rgb]{0.73,0.13,0.13}{##1}}}
\expandafter\def\csname PY@tok@sb\endcsname{\def\PY@tc##1{\textcolor[rgb]{0.73,0.13,0.13}{##1}}}
\expandafter\def\csname PY@tok@sc\endcsname{\def\PY@tc##1{\textcolor[rgb]{0.73,0.13,0.13}{##1}}}
\expandafter\def\csname PY@tok@dl\endcsname{\def\PY@tc##1{\textcolor[rgb]{0.73,0.13,0.13}{##1}}}
\expandafter\def\csname PY@tok@s2\endcsname{\def\PY@tc##1{\textcolor[rgb]{0.73,0.13,0.13}{##1}}}
\expandafter\def\csname PY@tok@sh\endcsname{\def\PY@tc##1{\textcolor[rgb]{0.73,0.13,0.13}{##1}}}
\expandafter\def\csname PY@tok@s1\endcsname{\def\PY@tc##1{\textcolor[rgb]{0.73,0.13,0.13}{##1}}}
\expandafter\def\csname PY@tok@mb\endcsname{\def\PY@tc##1{\textcolor[rgb]{0.40,0.40,0.40}{##1}}}
\expandafter\def\csname PY@tok@mf\endcsname{\def\PY@tc##1{\textcolor[rgb]{0.40,0.40,0.40}{##1}}}
\expandafter\def\csname PY@tok@mh\endcsname{\def\PY@tc##1{\textcolor[rgb]{0.40,0.40,0.40}{##1}}}
\expandafter\def\csname PY@tok@mi\endcsname{\def\PY@tc##1{\textcolor[rgb]{0.40,0.40,0.40}{##1}}}
\expandafter\def\csname PY@tok@il\endcsname{\def\PY@tc##1{\textcolor[rgb]{0.40,0.40,0.40}{##1}}}
\expandafter\def\csname PY@tok@mo\endcsname{\def\PY@tc##1{\textcolor[rgb]{0.40,0.40,0.40}{##1}}}
\expandafter\def\csname PY@tok@ch\endcsname{\let\PY@it=\textit\def\PY@tc##1{\textcolor[rgb]{0.25,0.50,0.50}{##1}}}
\expandafter\def\csname PY@tok@cm\endcsname{\let\PY@it=\textit\def\PY@tc##1{\textcolor[rgb]{0.25,0.50,0.50}{##1}}}
\expandafter\def\csname PY@tok@cpf\endcsname{\let\PY@it=\textit\def\PY@tc##1{\textcolor[rgb]{0.25,0.50,0.50}{##1}}}
\expandafter\def\csname PY@tok@c1\endcsname{\let\PY@it=\textit\def\PY@tc##1{\textcolor[rgb]{0.25,0.50,0.50}{##1}}}
\expandafter\def\csname PY@tok@cs\endcsname{\let\PY@it=\textit\def\PY@tc##1{\textcolor[rgb]{0.25,0.50,0.50}{##1}}}

\def\PYZbs{\char`\\}
\def\PYZus{\char`\_}
\def\PYZob{\char`\{}
\def\PYZcb{\char`\}}
\def\PYZca{\char`\^}
\def\PYZam{\char`\&}
\def\PYZlt{\char`\<}
\def\PYZgt{\char`\>}
\def\PYZsh{\char`\#}
\def\PYZpc{\char`\%}
\def\PYZdl{\char`\$}
\def\PYZhy{\char`\-}
\def\PYZsq{\char`\'}
\def\PYZdq{\char`\"}
\def\PYZti{\char`\~}
% for compatibility with earlier versions
\def\PYZat{@}
\def\PYZlb{[}
\def\PYZrb{]}
\makeatother


    % Exact colors from NB
    \definecolor{incolor}{rgb}{0.0, 0.0, 0.5}
    \definecolor{outcolor}{rgb}{0.545, 0.0, 0.0}



    
    % Prevent overflowing lines due to hard-to-break entities
    \sloppy 
    % Setup hyperref package
    \hypersetup{
      breaklinks=true,  % so long urls are correctly broken across lines
      colorlinks=true,
      urlcolor=urlcolor,
      linkcolor=linkcolor,
      citecolor=citecolor,
      }
    % Slightly bigger margins than the latex defaults
    
    \geometry{verbose,tmargin=1in,bmargin=1in,lmargin=1in,rmargin=1in}
    
    

    \begin{document}
    
    
    \maketitle
    
    

    
    \hypertarget{explore-bike-share-data}{%
\subsubsection{Explore Bike Share Data}\label{explore-bike-share-data}}

For this project, your goal is to ask and answer three questions about
the available bikeshare data from Washington, Chicago, and New York.
This notebook can be submitted directly through the workspace when you
are confident in your results.

You will be graded against the project
\href{https://review.udacity.com/\#!/rubrics/2508/view}{Rubric} by a
mentor after you have submitted. To get you started, you can use the
template below, but feel free to be creative in your solutions!

    \begin{Verbatim}[commandchars=\\\{\}]
{\color{incolor}In [{\color{incolor}182}]:} \PY{c+c1}{\PYZsh{} Here I am importing all of the necessary libraries for the analysis:}
          
          \PY{n}{ny} \PY{o}{=} \PY{n+nf}{read.csv}\PY{p}{(}\PY{l+s}{\PYZsq{}}\PY{l+s}{new\PYZus{}york\PYZus{}city.csv\PYZsq{}}\PY{p}{)}
          \PY{n}{wash} \PY{o}{=} \PY{n+nf}{read.csv}\PY{p}{(}\PY{l+s}{\PYZsq{}}\PY{l+s}{washington.csv\PYZsq{}}\PY{p}{)}
          \PY{n}{chi} \PY{o}{=} \PY{n+nf}{read.csv}\PY{p}{(}\PY{l+s}{\PYZsq{}}\PY{l+s}{chicago.csv\PYZsq{}}\PY{p}{)}
          \PY{n+nf}{library}\PY{p}{(}\PY{n}{ggplot2}\PY{p}{)}
          \PY{n+nf}{library}\PY{p}{(}\PY{n}{tidyverse}\PY{p}{)}
\end{Verbatim}


    \begin{Verbatim}[commandchars=\\\{\}]
{\color{incolor}In [{\color{incolor}183}]:} \PY{c+c1}{\PYZsh{}Checking the data in the ny dataframe}
          \PY{n+nf}{head}\PY{p}{(}\PY{n}{ny}\PY{p}{)}
\end{Verbatim}


    \begin{tabular}{r|lllllllll}
 X & Start.Time & End.Time & Trip.Duration & Start.Station & End.Station & User.Type & Gender & Birth.Year\\
\hline
	 5688089                   & 2017-06-11 14:55:05       & 2017-06-11 15:08:21       &  795                      & Suffolk St \& Stanton St & W Broadway \& Spring St  & Subscriber                & Male                      & 1998                     \\
	 4096714                   & 2017-05-11 15:30:11       & 2017-05-11 15:41:43       &  692                      & Lexington Ave \& E 63 St & 1 Ave \& E 78 St         & Subscriber                & Male                      & 1981                     \\
	 2173887                   & 2017-03-29 13:26:26       & 2017-03-29 13:48:31       & 1325                      & 1 Pl \& Clinton St       & Henry St \& Degraw St    & Subscriber                & Male                      & 1987                     \\
	 3945638                   & 2017-05-08 19:47:18       & 2017-05-08 19:59:01       &  703                      & Barrow St \& Hudson St   & W 20 St \& 8 Ave         & Subscriber                & Female                    & 1986                     \\
	 6208972                   & 2017-06-21 07:49:16       & 2017-06-21 07:54:46       &  329                      & 1 Ave \& E 44 St         & E 53 St \& 3 Ave         & Subscriber                & Male                      & 1992                     \\
	 1285652                   & 2017-02-22 18:55:24       & 2017-02-22 19:12:03       &  998                      & State St \& Smith St     & Bond St \& Fulton St     & Subscriber                & Male                      & 1986                     \\
\end{tabular}


    
    \begin{Verbatim}[commandchars=\\\{\}]
{\color{incolor}In [{\color{incolor}184}]:} \PY{c+c1}{\PYZsh{}Checking the data in the wash dataframe}
          \PY{n+nf}{head}\PY{p}{(}\PY{n}{wash}\PY{p}{)}
\end{Verbatim}


    \begin{tabular}{r|lllllll}
 X & Start.Time & End.Time & Trip.Duration & Start.Station & End.Station & User.Type\\
\hline
	 1621326                                               & 2017-06-21 08:36:34                                   & 2017-06-21 08:44:43                                   &  489.066                                              & 14th \& Belmont St NW                                & 15th \& K St NW                                      & Subscriber                                           \\
	  482740                                               & 2017-03-11 10:40:00                                   & 2017-03-11 10:46:00                                   &  402.549                                              & Yuma St \& Tenley Circle NW                          & Connecticut Ave \& Yuma St NW                        & Subscriber                                           \\
	 1330037                                               & 2017-05-30 01:02:59                                   & 2017-05-30 01:13:37                                   &  637.251                                              & 17th St \& Massachusetts Ave NW                      & 5th \& K St NW                                       & Subscriber                                           \\
	  665458                                               & 2017-04-02 07:48:35                                   & 2017-04-02 08:19:03                                   & 1827.341                                              & Constitution Ave \& 2nd St NW/DOL                    & M St \& Pennsylvania Ave NW                          & Customer                                             \\
	 1481135                                               & 2017-06-10 08:36:28                                   & 2017-06-10 09:02:17                                   & 1549.427                                              & Henry Bacon Dr \& Lincoln Memorial Circle NW         & Maine Ave \& 7th St SW                               & Subscriber                                           \\
	 1148202                                               & 2017-05-14 07:18:18                                   & 2017-05-14 07:24:56                                   &  398.000                                              & 1st \& K St SE                                       & Eastern Market Metro / Pennsylvania Ave \& 7th St SE & Subscriber                                           \\
\end{tabular}


    
    \begin{Verbatim}[commandchars=\\\{\}]
{\color{incolor}In [{\color{incolor}190}]:} \PY{c+c1}{\PYZsh{}Checking the data in the chi dataframe}
          \PY{n+nf}{head}\PY{p}{(}\PY{n}{chi}\PY{p}{)}
\end{Verbatim}


    \begin{tabular}{r|lllllllll}
 X & Start.Time & End.Time & Trip.Duration & Start.Station & End.Station & User.Type & Gender & Birth.Year\\
\hline
	 1423854                         & 2017-06-23 15:09:32             & 2017-06-23 15:14:53             &  321                            & Wood St \& Hubbard St          & Damen Ave \& Chicago Ave       & Subscriber                      & Male                            & 1992                           \\
	  955915                        & 2017-05-25 18:19:03            & 2017-05-25 18:45:53            & 1610                           & Theater on the Lake            & Sheffield Ave \& Waveland Ave & Subscriber                     & Female                         & 1992                          \\
	    9031                         & 2017-01-04 08:27:49             & 2017-01-04 08:34:45             &  416                            & May St \& Taylor St            & Wood St \& Taylor St           & Subscriber                      & Male                            & 1981                           \\
	  304487                         & 2017-03-06 13:49:38             & 2017-03-06 13:55:28             &  350                            & Christiana Ave \& Lawrence Ave & St. Louis Ave \& Balmoral Ave  & Subscriber                      & Male                            & 1986                           \\
	   45207                         & 2017-01-17 14:53:07             & 2017-01-17 15:02:01             &  534                            & Clark St \& Randolph St        & Desplaines St \& Jackson Blvd  & Subscriber                      & Male                            & 1975                           \\
	 1473887                         & 2017-06-26 09:01:20             & 2017-06-26 09:11:06             &  586                            & Clinton St \& Washington Blvd  & Canal St \& Taylor St          & Subscriber                      & Male                            & 1990                           \\
\end{tabular}


    
    \hypertarget{data-cleaning-wrangling}{%
\subsubsection{Data Cleaning
(Wrangling)}\label{data-cleaning-wrangling}}

\textbf{In the following command boxes, I take a few steps to ensure my
data is ready for the analysis. I remove columns that I dont need,
validate the selection of the columns, ensure each table is labeled with
its appropriate location, merge all tables into one, and validate the
merge was successful}

    \begin{Verbatim}[commandchars=\\\{\}]
{\color{incolor}In [{\color{incolor}191}]:} \PY{c+c1}{\PYZsh{}Selecting only the columns that I will need to answer the questions}
          \PY{n}{ny.new}\PY{o}{\PYZlt{}\PYZhy{}}\PY{n}{ny}\PY{n+nf}{[c}\PY{p}{(}\PY{l+m}{2}\PY{o}{:}\PY{l+m}{7}\PY{p}{)}\PY{n}{]}
          \PY{n}{wash.new}\PY{o}{\PYZlt{}\PYZhy{}}\PY{n}{wash}\PY{n+nf}{[c}\PY{p}{(}\PY{l+m}{2}\PY{o}{:}\PY{l+m}{7}\PY{p}{)}\PY{n}{]}
          \PY{n}{chi.new}\PY{o}{\PYZlt{}\PYZhy{}}\PY{n}{chi}\PY{n+nf}{[c}\PY{p}{(}\PY{l+m}{2}\PY{o}{:}\PY{l+m}{7}\PY{p}{)}\PY{n}{]}
\end{Verbatim}


    \begin{Verbatim}[commandchars=\\\{\}]
{\color{incolor}In [{\color{incolor}192}]:} \PY{c+c1}{\PYZsh{}Validating if the columns were selected correctly for each dataset}
          \PY{n+nf}{names}\PY{p}{(}\PY{n}{ny.new}\PY{p}{)}
          \PY{n+nf}{names}\PY{p}{(}\PY{n}{wash.new}\PY{p}{)}
          \PY{n+nf}{names}\PY{p}{(}\PY{n}{chi.new}\PY{p}{)}
\end{Verbatim}


    \begin{enumerate*}
\item 'Start.Time'
\item 'End.Time'
\item 'Trip.Duration'
\item 'Start.Station'
\item 'End.Station'
\item 'User.Type'
\end{enumerate*}


    
    \begin{enumerate*}
\item 'Start.Time'
\item 'End.Time'
\item 'Trip.Duration'
\item 'Start.Station'
\item 'End.Station'
\item 'User.Type'
\end{enumerate*}


    
    \begin{enumerate*}
\item 'Start.Time'
\item 'End.Time'
\item 'Trip.Duration'
\item 'Start.Station'
\item 'End.Station'
\item 'User.Type'
\end{enumerate*}


    
    \begin{Verbatim}[commandchars=\\\{\}]
{\color{incolor}In [{\color{incolor}193}]:} \PY{c+c1}{\PYZsh{}labeling the location prior to merge}
          \PY{n}{wash.new}\PY{o}{\PYZdl{}}\PY{n}{location} \PY{o}{\PYZlt{}\PYZhy{}} \PY{l+s}{\PYZsq{}}\PY{l+s}{DC\PYZsq{}}
          \PY{n}{ny.new}\PY{o}{\PYZdl{}}\PY{n}{location} \PY{o}{\PYZlt{}\PYZhy{}} \PY{l+s}{\PYZsq{}}\PY{l+s}{NYC\PYZsq{}}
          \PY{n}{chi.new}\PY{o}{\PYZdl{}}\PY{n}{location} \PY{o}{\PYZlt{}\PYZhy{}} \PY{l+s}{\PYZsq{}}\PY{l+s}{CHI\PYZsq{}}
\end{Verbatim}


    \begin{Verbatim}[commandchars=\\\{\}]
{\color{incolor}In [{\color{incolor}194}]:} \PY{c+c1}{\PYZsh{}merge datasets together into one dataset}
          \PY{n}{mainList} \PY{o}{\PYZlt{}\PYZhy{}}\PY{n+nf}{rbind}\PY{p}{(}\PY{n}{wash.new}\PY{p}{,}\PY{n}{ny.new}\PY{p}{,}\PY{n}{chi.new}\PY{p}{)}
\end{Verbatim}


    \begin{Verbatim}[commandchars=\\\{\}]
{\color{incolor}In [{\color{incolor}195}]:} \PY{c+c1}{\PYZsh{}Validating the merge was successful by showing the top rows of the new dataframe}
          \PY{n+nf}{head}\PY{p}{(}\PY{n}{mainList}\PY{p}{)}
\end{Verbatim}


    \begin{tabular}{r|lllllll}
 Start.Time & End.Time & Trip.Duration & Start.Station & End.Station & User.Type & location\\
\hline
	 2017-06-21 08:36:34                                   & 2017-06-21 08:44:43                                   &  489.066                                              & 14th \& Belmont St NW                                & 15th \& K St NW                                      & Subscriber                                            & DC                                                   \\
	 2017-03-11 10:40:00                                   & 2017-03-11 10:46:00                                   &  402.549                                              & Yuma St \& Tenley Circle NW                          & Connecticut Ave \& Yuma St NW                        & Subscriber                                            & DC                                                   \\
	 2017-05-30 01:02:59                                   & 2017-05-30 01:13:37                                   &  637.251                                              & 17th St \& Massachusetts Ave NW                      & 5th \& K St NW                                       & Subscriber                                            & DC                                                   \\
	 2017-04-02 07:48:35                                   & 2017-04-02 08:19:03                                   & 1827.341                                              & Constitution Ave \& 2nd St NW/DOL                    & M St \& Pennsylvania Ave NW                          & Customer                                              & DC                                                   \\
	 2017-06-10 08:36:28                                   & 2017-06-10 09:02:17                                   & 1549.427                                              & Henry Bacon Dr \& Lincoln Memorial Circle NW         & Maine Ave \& 7th St SW                               & Subscriber                                            & DC                                                   \\
	 2017-05-14 07:18:18                                   & 2017-05-14 07:24:56                                   &  398.000                                              & 1st \& K St SE                                       & Eastern Market Metro / Pennsylvania Ave \& 7th St SE & Subscriber                                            & DC                                                   \\
\end{tabular}


    
    \hypertarget{question-1}{%
\subsubsection{Question 1}\label{question-1}}

\textbf{What is the percentage of subcribers from the entire population}

    \begin{Verbatim}[commandchars=\\\{\}]
{\color{incolor}In [{\color{incolor}196}]:} \PY{c+c1}{\PYZsh{}in this command I create a dataframe to store the aggregate results}
          \PY{n}{aggList} \PY{o}{=} \PY{n+nf}{aggregate}\PY{p}{(}\PY{n}{location}\PY{o}{\PYZti{}}\PY{n}{User.Type}\PY{p}{,}\PY{n}{mainList}\PY{p}{,}\PY{n}{FUN} \PY{o}{=} \PY{n}{length}\PY{p}{)}
          \PY{n+nf}{colnames}\PY{p}{(}\PY{n}{aggList}\PY{p}{)}\PY{o}{\PYZlt{}\PYZhy{}} \PY{n+nf}{c}\PY{p}{(}\PY{l+s}{\PYZdq{}}\PY{l+s}{User\PYZus{}Type\PYZdq{}}\PY{p}{,}\PY{l+s}{\PYZdq{}}\PY{l+s}{User\PYZus{}Count\PYZdq{}}\PY{p}{)}
          \PY{c+c1}{\PYZsh{}Below I create a percentage column for each user type}
          \PY{n}{aggList}\PY{o}{\PYZdl{}}\PY{n}{Percentage} \PY{o}{=} \PY{n}{aggList}\PY{o}{\PYZdl{}}\PY{n}{User\PYZus{}Count}\PY{o}{/}\PY{n+nf}{sum}\PY{p}{(}\PY{n}{aggList}\PY{o}{\PYZdl{}}\PY{n}{User\PYZus{}Count}\PY{p}{)}
          \PY{n}{aggList}\PY{o}{\PYZdl{}}\PY{n}{Percentage} \PY{o}{=} \PY{n}{aggList}\PY{o}{\PYZdl{}}\PY{n}{Percentage} \PY{o}{*} \PY{l+m}{100}
          \PY{n}{aggList}\PY{o}{\PYZdl{}}\PY{n}{Percentage} \PY{o}{=} \PY{n+nf}{round}\PY{p}{(}\PY{n}{aggList}\PY{o}{\PYZdl{}}\PY{n}{Percentage}\PY{p}{,}\PY{n}{digits} \PY{o}{=} \PY{l+m}{2}\PY{p}{)}
\end{Verbatim}


    \begin{Verbatim}[commandchars=\\\{\}]
{\color{incolor}In [{\color{incolor}197}]:} \PY{c+c1}{\PYZsh{}Showing the new datafram to ensure it was created correctly}
          \PY{n+nf}{print}\PY{p}{(}\PY{n}{aggList}\PY{p}{)}
\end{Verbatim}


    \begin{Verbatim}[commandchars=\\\{\}]
   User\_Type User\_Count Percentage
1                   121       0.08
2   Customer      30754      20.17
3 Subscriber     121576      79.75

    \end{Verbatim}

    \begin{Verbatim}[commandchars=\\\{\}]
{\color{incolor}In [{\color{incolor}198}]:} \PY{c+c1}{\PYZsh{}In this command I create a Pie Chart to show the percentages for each User Type in a graphical form}
          \PY{n}{aggListlab} \PY{o}{\PYZlt{}\PYZhy{}} \PY{n}{aggList}
          \PY{n}{pieChart} \PY{o}{\PYZlt{}\PYZhy{}} \PY{n+nf}{ggplot}\PY{p}{(}\PY{n}{data} \PY{o}{=} \PY{n}{aggListlab}\PY{p}{,} \PY{n+nf}{aes}\PY{p}{(}\PY{n}{x}\PY{o}{=} \PY{l+s}{\PYZsq{}}\PY{l+s}{\PYZsq{}}\PY{p}{,}\PY{n}{y} \PY{o}{=} \PY{n}{User\PYZus{}Count}\PY{p}{,} \PY{n}{fill} \PY{o}{=} \PY{n+nf}{paste0}\PY{p}{(}\PY{n}{User\PYZus{}Type}\PY{p}{,}\PY{l+s}{\PYZsq{}}\PY{l+s}{(\PYZsq{}}\PY{p}{,} \PY{n+nf}{round}\PY{p}{(}\PY{n}{User\PYZus{}Count}\PY{o}{/}\PY{n+nf}{sum}\PY{p}{(}\PY{n}{User\PYZus{}Count}\PY{p}{)}\PY{o}{*}\PY{l+m}{100}\PY{p}{,}\PY{n}{digits} \PY{o}{=} \PY{l+m}{2}\PY{p}{)}\PY{p}{,}\PY{l+s}{\PYZsq{}}\PY{l+s}{\PYZpc{})\PYZsq{}} \PY{p}{)}\PY{p}{)}\PY{p}{)}
          \PY{n}{pieChart} \PY{o}{\PYZlt{}\PYZhy{}} \PY{n}{pieChart} \PY{o}{+} \PY{n+nf}{geom\PYZus{}bar}\PY{p}{(}\PY{n}{width} \PY{o}{=} \PY{l+m}{1}\PY{p}{,} \PY{n}{stat} \PY{o}{=} \PY{l+s}{\PYZdq{}}\PY{l+s}{identity\PYZdq{}}\PY{p}{)}
          \PY{n}{pieChart} \PY{o}{\PYZlt{}\PYZhy{}} \PY{n}{pieChart} \PY{o}{+} \PY{n+nf}{geom\PYZus{}text}\PY{p}{(}\PY{n+nf}{aes}\PY{p}{(}\PY{n}{x} \PY{o}{=} \PY{l+m}{1.4}\PY{p}{,}\PY{n}{label} \PY{o}{=} \PY{n}{User\PYZus{}Type} \PY{p}{)}\PY{p}{,} \PY{n}{position} \PY{o}{=} \PY{n+nf}{position\PYZus{}stack}\PY{p}{(}\PY{n}{vjust} \PY{o}{=} \PY{l+m}{0.7}\PY{p}{)}\PY{p}{)}
          \PY{n}{pieChart} \PY{o}{\PYZlt{}\PYZhy{}} \PY{n}{pieChart} \PY{o}{+} \PY{n+nf}{theme\PYZus{}void}\PY{p}{(}\PY{p}{)}
          \PY{n}{pieChart} \PY{o}{\PYZlt{}\PYZhy{}} \PY{n}{pieChart} \PY{o}{+} \PY{n+nf}{theme\PYZus{}classic}\PY{p}{(}\PY{p}{)}
          \PY{n}{pieChart} \PY{o}{\PYZlt{}\PYZhy{}} \PY{n}{pieChart} \PY{o}{+} \PY{n+nf}{theme}\PY{p}{(}\PY{n}{legend.position} \PY{o}{=} \PY{l+s}{\PYZdq{}}\PY{l+s}{bottom\PYZdq{}}\PY{p}{)}
          \PY{n}{pieChart}  \PY{o}{\PYZlt{}\PYZhy{}} \PY{n}{pieChart} \PY{o}{+} \PY{n+nf}{coord\PYZus{}polar}\PY{p}{(}\PY{l+s}{\PYZdq{}}\PY{l+s}{y\PYZdq{}}\PY{p}{,} \PY{n}{start}\PY{o}{=}\PY{l+m}{0}\PY{p}{)}
          \PY{n}{pieChart} \PY{o}{\PYZlt{}\PYZhy{}} \PY{n}{pieChart} \PY{o}{+}   \PY{n+nf}{theme}\PY{p}{(}\PY{n}{axis.line} \PY{o}{=} \PY{n+nf}{element\PYZus{}blank}\PY{p}{(}\PY{p}{)}\PY{p}{)}
          \PY{n}{pieChart} \PY{o}{\PYZlt{}\PYZhy{}} \PY{n}{pieChart} \PY{o}{+}   \PY{n+nf}{theme}\PY{p}{(}\PY{n}{axis.text} \PY{o}{=} \PY{n+nf}{element\PYZus{}blank}\PY{p}{(}\PY{p}{)}\PY{p}{)}
          \PY{n}{pieChart} \PY{o}{\PYZlt{}\PYZhy{}} \PY{n}{pieChart} \PY{o}{+}   \PY{n+nf}{theme}\PY{p}{(}\PY{n}{axis.ticks} \PY{o}{=} \PY{n+nf}{element\PYZus{}blank}\PY{p}{(}\PY{p}{)}\PY{p}{)}
          \PY{n}{pieChart} \PY{o}{\PYZlt{}\PYZhy{}} \PY{n}{pieChart} \PY{o}{+}   \PY{n+nf}{labs}\PY{p}{(}\PY{n}{x} \PY{o}{=} \PY{k+kc}{NULL}\PY{p}{,} \PY{n}{y} \PY{o}{=} \PY{k+kc}{NULL}\PY{p}{,} \PY{n}{fill} \PY{o}{=} \PY{k+kc}{NULL}\PY{p}{)}
          \PY{n}{pieChart} \PY{o}{\PYZlt{}\PYZhy{}} \PY{n}{pieChart} \PY{o}{+} \PY{n+nf}{labs}\PY{p}{(}\PY{n}{title} \PY{o}{=}\PY{l+s}{\PYZdq{}}\PY{l+s}{Pie chart of User Type for entire population\PYZdq{}}\PY{p}{)}
          \PY{n}{pieChart}
\end{Verbatim}


    \begin{center}
    \adjustimage{max size={0.9\linewidth}{0.9\paperheight}}{output_14_0.png}
    \end{center}
    { \hspace*{\fill} \\}
    
    \textbf{Summary of your question 1 results goes here}
\textgreater{}Based on the results, \textbf{subscribers consist of
79.75\% of the total population.} The pie chart clearly show the
dominance of the subscriber population compared to the remainder

\begin{quote}
Additional insights: 20.17\% of the population were categorized as
``Customers'' and the rest had no categorization (0.08\%) . This can
signify a failure in categorization for a small subsection of the
population and should be further investigated
\end{quote}

    \hypertarget{question-2}{%
\subsubsection{Question 2}\label{question-2}}

\textbf{What are the average trip durations for each user type by
location}

    \begin{Verbatim}[commandchars=\\\{\}]
{\color{incolor}In [{\color{incolor}199}]:} \PY{c+c1}{\PYZsh{}in this command I created a dataframe to store the aggregate results}
          \PY{n}{aggList} \PY{o}{=} \PY{n+nf}{aggregate}\PY{p}{(}\PY{n}{Trip.Duration}\PY{o}{\PYZti{}}\PY{n}{User.Type}\PY{o}{+}\PY{n}{location}\PY{p}{,}\PY{n}{mainList}\PY{p}{,}\PY{n}{mean}\PY{p}{)}
          \PY{n+nf}{colnames}\PY{p}{(}\PY{n}{aggList}\PY{p}{)}\PY{o}{\PYZlt{}\PYZhy{}} \PY{n+nf}{c}\PY{p}{(}\PY{l+s}{\PYZdq{}}\PY{l+s}{User\PYZus{}Type\PYZdq{}}\PY{p}{,}\PY{l+s}{\PYZdq{}}\PY{l+s}{Location\PYZdq{}}\PY{p}{,}\PY{l+s}{\PYZdq{}}\PY{l+s}{Average\PYZus{}Time\PYZdq{}}\PY{p}{)}
\end{Verbatim}


    \begin{Verbatim}[commandchars=\\\{\}]
{\color{incolor}In [{\color{incolor}200}]:} \PY{c+c1}{\PYZsh{}Here I am printing the list to ensure it came out correct}
          \PY{n+nf}{print}\PY{p}{(}\PY{n}{aggList}\PY{p}{)}
\end{Verbatim}


    \begin{Verbatim}[commandchars=\\\{\}]
   User\_Type Location Average\_Time
1                 CHI    3020.0000
2   Customer      CHI    1929.9771
3 Subscriber      CHI     685.0270
4   Customer       DC    2634.4289
5 Subscriber       DC     733.3260
6                 NYC    1838.4915
7   Customer      NYC    2193.0759
8 Subscriber      NYC     755.3829

    \end{Verbatim}

    \begin{Verbatim}[commandchars=\\\{\}]
{\color{incolor}In [{\color{incolor}201}]:} \PY{c+c1}{\PYZsh{}During Question 1\PYZsq{}s excersize, I found out the blank user\PYZus{}type entries were outliers, with only 0.08\PYZpc{} of the entire dataset being unlabeled. So for better charting, I remove those from the dataframe:}
          
          \PY{n}{aggListFiltered} \PY{o}{\PYZlt{}\PYZhy{}} \PY{n}{aggList}\PY{n}{[aggList}\PY{o}{\PYZdl{}}\PY{n}{User\PYZus{}Type} \PY{o}{!=} \PY{l+s}{\PYZdq{}}\PY{l+s}{\PYZdq{}}\PY{p}{,}\PY{n}{]}
          \PY{n+nf}{print}\PY{p}{(}\PY{n}{aggListFiltered}\PY{p}{)}
\end{Verbatim}


    \begin{Verbatim}[commandchars=\\\{\}]
   User\_Type Location Average\_Time
2   Customer      CHI    1929.9771
3 Subscriber      CHI     685.0270
4   Customer       DC    2634.4289
5 Subscriber       DC     733.3260
7   Customer      NYC    2193.0759
8 Subscriber      NYC     755.3829

    \end{Verbatim}

    \begin{Verbatim}[commandchars=\\\{\}]
{\color{incolor}In [{\color{incolor}202}]:} \PY{c+c1}{\PYZsh{} In this command box, I am creating the bar graph. I decided to flip the axis as it is better to compare average times with}
          
          \PY{n}{bre} \PY{o}{\PYZlt{}\PYZhy{}} \PY{n+nf}{c}\PY{p}{(}\PY{n+nf}{seq}\PY{p}{(}\PY{l+m}{\PYZhy{}3000}\PY{p}{,} \PY{l+m}{3000}\PY{p}{,} \PY{n}{by} \PY{o}{=} \PY{l+m}{1000}\PY{p}{)}\PY{p}{)}
          \PY{n}{labl} \PY{o}{=} \PY{n+nf}{c}\PY{p}{(}\PY{n+nf}{seq}\PY{p}{(}\PY{l+m}{3000}\PY{p}{,} \PY{l+m}{0}\PY{p}{,} \PY{l+m}{\PYZhy{}1000}\PY{p}{)}\PY{p}{,} \PY{n+nf}{seq}\PY{p}{(}\PY{l+m}{1000}\PY{p}{,} \PY{l+m}{3000}\PY{p}{,} \PY{l+m}{1000}\PY{p}{)}\PY{p}{)}
          \PY{n}{barplt} \PY{o}{\PYZlt{}\PYZhy{}} \PY{n}{aggListFiltered} \PY{o}{\PYZpc{}\PYZgt{}\PYZpc{}}  \PY{c+c1}{\PYZsh{} Cast the users table as a number}
          \PY{n+nf}{mutate}\PY{p}{(}\PY{n}{Average\PYZus{}Time} \PY{o}{=} \PY{n+nf}{as.numeric}\PY{p}{(}\PY{n}{Average\PYZus{}Time}\PY{p}{)}\PY{p}{)} \PY{o}{\PYZpc{}\PYZgt{}\PYZpc{}}
          \PY{n+nf}{ggplot}\PY{p}{(}\PY{n+nf}{aes}\PY{p}{(}\PY{n}{Location}\PY{p}{,} \PY{n}{y} \PY{o}{=} \PY{n}{Average\PYZus{}Time}\PY{p}{,} \PY{n}{fill} \PY{o}{=} \PY{n}{User\PYZus{}Type}\PY{p}{,} \PY{p}{)}\PY{p}{)} \PY{o}{+}
          \PY{n+nf}{geom\PYZus{}col}\PY{p}{(}\PY{n}{position} \PY{o}{=} \PY{l+s}{\PYZdq{}}\PY{l+s}{dodge\PYZdq{}}\PY{p}{)} \PY{o}{+}
          \PY{n+nf}{scale\PYZus{}y\PYZus{}continuous}\PY{p}{(}\PY{n}{breaks} \PY{o}{=} \PY{n}{bre}\PY{p}{,} \PY{n}{labels} \PY{o}{=} \PY{n}{labl}\PY{p}{)} \PY{o}{+}
          \PY{n+nf}{coord\PYZus{}flip}\PY{p}{(}\PY{p}{)} \PY{o}{+}
          \PY{n+nf}{labs}\PY{p}{(}\PY{n}{title}\PY{o}{=}\PY{l+s}{\PYZdq{}}\PY{l+s}{Average Trip Duration\PYZdq{}}\PY{p}{)} \PY{o}{+}
          \PY{n+nf}{theme}\PY{p}{(}\PY{n}{plot.title} \PY{o}{=} \PY{n+nf}{element\PYZus{}text}\PY{p}{(}\PY{n}{hjust} \PY{o}{=} \PY{l+m}{.5}\PY{p}{)}\PY{p}{,}
          \PY{n}{axis.ticks} \PY{o}{=} \PY{n+nf}{element\PYZus{}blank}\PY{p}{(}\PY{p}{)}\PY{p}{)}
          \PY{n}{barplt}
\end{Verbatim}


    \begin{center}
    \adjustimage{max size={0.9\linewidth}{0.9\paperheight}}{output_20_0.png}
    \end{center}
    { \hspace*{\fill} \\}
    
    \textbf{Summary of your question 2 results goes here.}

\begin{quote}
The average trip duration are as follows:
\end{quote}

\begin{quote}
Chicago Customers: 1929
\end{quote}

\begin{quote}
Chicago Subscribers : 685
\end{quote}

\begin{quote}
DC Customers: 2634
\end{quote}

\begin{quote}
DC Subscribers : 733
\end{quote}

\begin{quote}
NYC Customers: 2193
\end{quote}

\begin{quote}
NYC SUbscribers: 755
\end{quote}

It is clear that Customers hold a much higher average trip duration
compared to subscribers on every city. A root cause analysis is
warranted to determine the main reason why customers utilize the bikes
more. The learning lessons can be apply to assist in increasing the trip
time of the subscribers to increase revenue.

    \hypertarget{question-3}{%
\subsubsection{Question 3}\label{question-3}}

\textbf{What location shares the most amount of bikes}

    \begin{Verbatim}[commandchars=\\\{\}]
{\color{incolor}In [{\color{incolor}207}]:} \PY{c+c1}{\PYZsh{}Here I am creating a list out of the aggregation of city bike count:}
          \PY{n}{aggList} \PY{o}{=} \PY{n+nf}{aggregate}\PY{p}{(}\PY{n}{User.Type}\PY{o}{\PYZti{}}\PY{n}{location}\PY{p}{,}\PY{n}{mainList}\PY{p}{,}\PY{n}{FUN} \PY{o}{=} \PY{n}{length}\PY{p}{)}
          \PY{n+nf}{colnames}\PY{p}{(}\PY{n}{aggList}\PY{p}{)}\PY{o}{\PYZlt{}\PYZhy{}} \PY{n+nf}{c}\PY{p}{(}\PY{l+s}{\PYZdq{}}\PY{l+s}{Location\PYZdq{}}\PY{p}{,}\PY{l+s}{\PYZdq{}}\PY{l+s}{Total\PYZus{}Count\PYZdq{}}\PY{p}{)}
\end{Verbatim}


    \begin{Verbatim}[commandchars=\\\{\}]
{\color{incolor}In [{\color{incolor}208}]:} \PY{c+c1}{\PYZsh{}I printed the list to see if the aggregation worked}
          
          \PY{n}{aggList}
\end{Verbatim}


    \begin{tabular}{r|ll}
 Location & Total\_Count\\
\hline
	 CHI   &  8630\\
	 DC    & 89051\\
	 NYC   & 54770\\
\end{tabular}


    
    \begin{Verbatim}[commandchars=\\\{\}]
{\color{incolor}In [{\color{incolor}250}]:} \PY{c+c1}{\PYZsh{}Here I am creating the stacked bar chart:}
          
          \PY{n}{aggListlab} \PY{o}{\PYZlt{}\PYZhy{}} \PY{n}{aggList}
          \PY{n}{stackedChart} \PY{o}{\PYZlt{}\PYZhy{}} \PY{n+nf}{ggplot}\PY{p}{(}\PY{n}{data} \PY{o}{=} \PY{n}{aggListlab}\PY{p}{,} \PY{n+nf}{aes}\PY{p}{(}\PY{n}{x}\PY{o}{=} \PY{l+s}{\PYZsq{}}\PY{l+s}{\PYZsq{}}\PY{p}{,}\PY{n}{y} \PY{o}{=} \PY{n}{Total\PYZus{}Count}\PY{p}{,} \PY{n}{fill} \PY{o}{=} \PY{n}{Location}\PY{p}{)}\PY{p}{)}
          \PY{n}{stackedChart} \PY{o}{\PYZlt{}\PYZhy{}} \PY{n}{stackedChart} \PY{o}{+} \PY{n+nf}{geom\PYZus{}bar}\PY{p}{(}\PY{n}{width} \PY{o}{=} \PY{l+m}{.7}\PY{p}{,} \PY{n}{stat} \PY{o}{=} \PY{l+s}{\PYZdq{}}\PY{l+s}{identity\PYZdq{}}\PY{p}{)}
          \PY{n}{stackedChart} \PY{o}{\PYZlt{}\PYZhy{}} \PY{n}{stackedChart} \PY{o}{+} \PY{n+nf}{geom\PYZus{}text}\PY{p}{(}\PY{n+nf}{aes}\PY{p}{(}\PY{n}{label} \PY{o}{=} \PY{n}{Total\PYZus{}Count} \PY{p}{)}\PY{p}{,} \PY{n}{position} \PY{o}{=} \PY{n+nf}{position\PYZus{}stack}\PY{p}{(}\PY{n}{vjust} \PY{o}{=} \PY{l+m}{0.7}\PY{p}{)}\PY{p}{)}
          \PY{n}{stackedChart} \PY{o}{\PYZlt{}\PYZhy{}} \PY{n}{stackedChart} \PY{o}{+} \PY{n+nf}{theme\PYZus{}void}\PY{p}{(}\PY{p}{)}
          \PY{n}{stackedChart} \PY{o}{\PYZlt{}\PYZhy{}} \PY{n}{stackedChart} \PY{o}{+} \PY{n+nf}{theme\PYZus{}classic}\PY{p}{(}\PY{p}{)}
          \PY{n}{stackedChart} \PY{o}{\PYZlt{}\PYZhy{}} \PY{n}{stackedChart} \PY{o}{+} \PY{n+nf}{theme}\PY{p}{(}\PY{n}{legend.position} \PY{o}{=} \PY{l+s}{\PYZdq{}}\PY{l+s}{left\PYZdq{}}\PY{p}{)}
          \PY{n}{stackedChart}
\end{Verbatim}


    \begin{center}
    \adjustimage{max size={0.9\linewidth}{0.9\paperheight}}{output_25_0.png}
    \end{center}
    { \hspace*{\fill} \\}
    
    \begin{Verbatim}[commandchars=\\\{\}]
{\color{incolor}In [{\color{incolor}240}]:} \PY{c+c1}{\PYZsh{}Here I used different math functions to analyze the data in different ways}
          \PY{n+nf}{print}\PY{p}{(}\PY{l+s}{\PYZdq{}}\PY{l+s}{Mean of Total Count\PYZdq{}}\PY{p}{,}\PY{p}{)}
          \PY{n+nf}{mean}\PY{p}{(}\PY{n}{aggList}\PY{o}{\PYZdl{}}\PY{n}{Total\PYZus{}Count}\PY{p}{)}
          \PY{n+nf}{print}\PY{p}{(}\PY{l+s}{\PYZdq{}}\PY{l+s}{Median of Total Count\PYZdq{}}\PY{p}{,}\PY{p}{)}
          \PY{n+nf}{median}\PY{p}{(}\PY{n}{aggList}\PY{o}{\PYZdl{}}\PY{n}{Total\PYZus{}Count}\PY{p}{)}
          \PY{n+nf}{print}\PY{p}{(}\PY{l+s}{\PYZdq{}}\PY{l+s}{Standard Deviation of Total Count\PYZdq{}}\PY{p}{,}\PY{p}{)}
          \PY{n+nf}{sd}\PY{p}{(}\PY{n}{aggList}\PY{o}{\PYZdl{}}\PY{n}{Total\PYZus{}Count}\PY{p}{)}
          
          \PY{c+c1}{\PYZsh{}Here I printed the final results list}
          \PY{n+nf}{print}\PY{p}{(}\PY{l+s}{\PYZdq{}}\PY{l+s}{List of Total Count\PYZdq{}}\PY{p}{,}\PY{p}{)}
          \PY{n+nf}{print}\PY{p}{(}\PY{n}{aggList}\PY{p}{)}
\end{Verbatim}


    \begin{Verbatim}[commandchars=\\\{\}]
[1] "Mean of Total Count"

    \end{Verbatim}

    50817

    
    \begin{Verbatim}[commandchars=\\\{\}]
[1] "Median of Total Count"

    \end{Verbatim}

    54770

    
    \begin{Verbatim}[commandchars=\\\{\}]
[1] "Standard Deviation of Total Count"

    \end{Verbatim}

    40355.9656928192

    
    \begin{Verbatim}[commandchars=\\\{\}]
[1] "List of Total Count"
  Location Total\_Count
1      CHI        8630
2       DC       89051
3      NYC       54770

    \end{Verbatim}

    \textbf{Summary of your question 3 results goes here.}

The location that shares the most amount is DC at 89,051. NYC comes at a
second place at 54,770. Chicaco comes at third place at 8,630 bikes. DC
bike sharing market should be further studied to determine why they are
being more successful at renting out bikes. The lessons can be used to
expand the markets of NYC and Chicago.

    \hypertarget{finishing-up}{%
\subsection{Finishing Up}\label{finishing-up}}

\begin{quote}
Congratulations! You have reached the end of the Explore Bikeshare Data
Project. You should be very proud of all you have accomplished!
\end{quote}

\begin{quote}
\textbf{Tip}: Once you are satisfied with your work here, check over
your report to make sure that it is satisfies all the areas of the
\href{https://review.udacity.com/\#!/rubrics/2508/view}{rubric}.
\end{quote}

\hypertarget{directions-to-submit}{%
\subsection{Directions to Submit}\label{directions-to-submit}}

\begin{quote}
Before you submit your project, you need to create a .html or .pdf
version of this notebook in the workspace here. To do that, run the code
cell below. If it worked correctly, you should get a return code of 0,
and you should see the generated .html file in the workspace directory
(click on the orange Jupyter icon in the upper left).
\end{quote}

\begin{quote}
Alternatively, you can download this report as .html via the
\textbf{File} \textgreater{} \textbf{Download as} submenu, and then
manually upload it into the workspace directory by clicking on the
orange Jupyter icon in the upper left, then using the Upload button.
\end{quote}

\begin{quote}
Once you've done this, you can submit your project by clicking on the
``Submit Project'' button in the lower right here. This will create and
submit a zip file with this .ipynb doc and the .html or .pdf version you
created. Congratulations!
\end{quote}

    \begin{Verbatim}[commandchars=\\\{\}]
{\color{incolor}In [{\color{incolor} }]:} \PY{n+nf}{system}\PY{p}{(}\PY{l+s}{\PYZsq{}}\PY{l+s}{python \PYZhy{}m nbconvert Explore\PYZus{}bikeshare\PYZus{}data.ipynb\PYZsq{}}\PY{p}{)}
\end{Verbatim}



    % Add a bibliography block to the postdoc
    
    
    
    \end{document}
